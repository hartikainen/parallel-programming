I took the old code from cp1, and changed it to work on GPU, using CUDA. Basically, the only thing I did was to move the matrix multiplication part of the task to GPU. This is implemented in matrix\_multiply -function, and called from the same correlate function as where it was done in cp1. correlate -function copies the matrix to be multiplied from the host memory in to the device memory, does the multiplication, and copies the results from the device to the host.

The benchmarks were run on the classroom computer 'Kiwi', and different block sizes (arbitrarily chosen to be powers of two) were tested. Benchmarks for are shown in the table \ref{tab:cp8-benchmarks}, and the corresponding visualizations in figure \ref{fig:figure/cp8}.

Fastest running time, 0.847s, was achieved on 8x8 block size.

%%% Local Variables:
%%% mode: latex
%%% TeX-master: "week4.tex"
%%% End:
