

\verbatimtable{benchmarks/cp1.txt}{Benchmarks for the exercise cp1}

\verbatimtable{benchmarks/cp2-1t.txt}{Benchmarks for the exercise cp2, using 1 thread}
\verbatimtable{benchmarks/cp2-2t.txt}{Benchmarks for the exercise cp2, using 2 thread}
\pagebreak

\verbatimtable{benchmarks/cp2-4t.txt}{Benchmarks for the exercise cp2, using 4 threads}
\verbatimtable{benchmarks/cp2-8t.txt}{Benchmarks for the exercise cp2, using 8 threads}

\verbatimtable{benchmarks/cp3-8t.txt}{Benchmarks for the exercise cp3, using 8 threads}
\pagebreak

\verbatimtable{benchmarks/cp4-2bs-8t.txt}{Benchmarks for the exercise cp4, using 8 threads, and 2*2 block size}
\verbatimtable{benchmarks/cp4-3bs-8t.txt}{Benchmarks for the exercise cp4, using 8 threads, and 3*3 block size}
\pagebreak

\figureFromFile{figure/cp1-mult-vs-time}{png}{Multiplication per second, using naive algorithm implemented in cp1}
\figureFromFile{figure/cp2-mult-vs-time}{png}{Multiplication per second, using multipthreaded algorithm implemented in cp2}
\figureFromFile{figure/cp3-mult-vs-time}{png}{Multiplication per second, using vectorized algorithm implemented in cp3}

\figureFromFile{figure/cp4-2bs-mult-vs-time}{png}{Multiplication per second, using cache blocking (block width 2) algorithm implemented in cp4}
\figureFromFile{figure/cp4-3bs-mult-vs-time}{png}{Multiplication per second, using cache blocking (block width 3) algorithm implemented in cp4}

\pagebreak
% \figureFromFile{grouped_runningtime_vs_cores}{png}{The number of the cores vs logarithmic running time, grouped by the input size}
% \pagebreak
% \figureFromFile{grouped_runningtime_vs_inputsize}{png}{Input size vs running time, grouped by the number of the cores}
% \pagebreak


%%% Local Variables:
%%% mode: latex
%%% TeX-master: "week2.tex"
%%% End:
